\begin{card}
	\frametitle{Übung 7: $\lambda$-Kalkül}
	\url{http://people.f4.htw-berlin.de/~hebold/htw/pka/exercises/algorithmen-lambdaCalculus_cont.pdf}
\end{card}

\begin{card}
	Definieren Sie im $\lambda$–Kalkül die
	\begin{enumerate}
	\item die Null (=0)
	\item die Nachfolgefunktion succ
	\item eine beliebige Zahl
	\end{enumerate}
	\hr
	\begin{enumerate}
	\item $^\lceil 0 ^\rceil = \lambda x.x$
	\item $^\lceil n+1 ^\rceil = \lambda x.((x \perp )\qquad ^\lceil n ^\rceil)$
	\item $^\lceil n ^\rceil = $beliebig häufig anwenden von 2.) ?
	\end{enumerate}
\end{card}

\begin{card}
	Definieren Sie im $\lambda$–Kalkül die primitiv rekursiven Funktionen: 
	\begin{enumerate}
	\item $0$
	\item $N$
	\item $P^m_n$
	\item $S^{n+1}$
	\end{enumerate}
\end{card}

\begin{card}
	Wie lautet der Fixpunkt von: 
	\begin{enumerate}
	\item not
	\item succ
	\end{enumerate}
\end{card}

\begin{card}
	Definieren Sie im $\lambda$-Kalkül das primitiv rekursive Funktionsschema	\textbf{R}
\end{card}

\begin{card}
	Beschreiben Sie den Unterschied von $=$ und $\equiv$ an einem Beispiel. 
\end{card}

\begin{card}
	Bei der Definition der Substitution werden Funktionen auf zwei Arten miteinander verknüpft. Nennen Sie die beiden Arten. Beschreiben Sie den Unterschied an einem Beispiel.
	\hr
\end{card}

\begin{card}
	Funktionen können durch Komposition oder Currying verknüpft werden.
	\begin{enumerate}
	\item Beschreiben Sie den Unterschied an einem Beispiel.
	\item Welche der beiden Verknüpfungsarten entspricht einer Funktionsdefinition in Java?
	\end{enumerate}
\end{card}