
\usepackage[german]{babel} % deutsch und deutsche Rechtschreibung
\usepackage[utf8]{inputenc}
\usepackage[T1]{fontenc} % Umlaute und deutsches Trennen
\usepackage{amsmath}
\usepackage{amsfonts}
\usepackage{amssymb}
\usepackage{mathabx}
\usepackage{graphicx}
\usepackage{url}
\usepackage{color}
\usepackage{listings}
\usepackage{multicol}
\usepackage{xspace}
\usepackage[normalem]{ulem}
\usepackage{wasysym}

\DeclareGraphicsExtensions{.pdf,.jpeg,.png}
\graphicspath{{img/}}

% Klammern {{{
\newcommand{\lrk}{\left(}
\newcommand{\rrk}{\right)}
\newcommand{\lgk}{\left\{}
\newcommand{\rgk}{\right\}}
\newcommand{\lek}{\left[}
\newcommand{\rek}{\right]}
% }}}

\newcommand{\field}[1]{\ensuremath{\mathbb{#1}}\xspace}
\newcommand{\N}{\field{N}}
\newcommand{\Z}{\field{Z}}
\newcommand{\Q}{\field{Q}}
\newcommand{\R}{\field{R}}
\newcommand{\C}{\field{C}}
\newcommand{\K}{\field{K}}

\newcommand{\inda}[2]{\textit{Induktionsanfang ($#1=#2$): }}
\newcommand{\indv}{\textit{Induktionsvoraussetzung ($\star$): }}
\newcommand{\inds}[1]{\textit{Induktionsschritt ($#1\rightsquigarrow #1+1$): }}
\newcommand{\proofend}{\begin{flushright}$\Box$\end{flushright}}

\newcommand{\BigO}[1]{\ensuremath{\operatorname{O}\bigl(#1\bigr)}}

\newcommand{\ent}{\mathrel{\widehat{=}}}

%\newcommand{\hr}{\noindent\makebox[\linewidth]{\rule{\paperwidth}{0.4pt}}\vspace{0.6em}\\}
%\newcommand{\newslide}{}

\newcommand{\hr}{\noindent\makebox[\linewidth]{\rule{\paperwidth}{0.4pt}}\vspace{0.6em}\\\pause}
\newcommand{\newslide}{\pause}


\newenvironment{card}[1]
  {\begin{frame}[fragile,environment=card]#1}
  {\end{frame}}
  
\lstset{
  inputencoding=utf8,
  basicstyle=\footnotesize\ttfamily,
  tabsize=2,
  breaklines=true,
  prebreak=\mbox{$\ \curvearrowright$},
  keepspaces=true,
  %columns=flexible,
  keywordstyle=\bfseries\color{blue},
  stringstyle=\color{red!40!black},
  commentstyle=\itshape\color{green!40!black},
  identifierstyle=\color{blue!40!black}
}
