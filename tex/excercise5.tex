\begin{card}
	\frametitle{Übung 5: Induktion}
	\url{http://people.f4.htw-berlin.de/~hebold/htw/pka/exercises/algorithmen-Induktion.pdf}
\end{card}

\begin{card}
  Beweisen Sie mit Hilfe der vollständigen Induktion:
  \begin{enumerate}[a)]
    \item $\forall n (n \in \N_0 \rightarrow 2^0 + 2^1 + \ldots + 2^n = 2^{n+1} - 1)$
    \item $n \in \N_0 \Rightarrow n^2 + n$ ist gerade
    \item $n \in \N_0 \Rightarrow 5^n + 7$ ist durch 4 teilbar
    \item $n \in \N_0 \Rightarrow 1 + 3 + 5 + \ldots + (2n - 1) = n^2$
    \item $n \in \N_0 \Rightarrow 3 \mid n^3 - n$
    \item $n \in \N_0 \Rightarrow 0 \cdot 0! + 1 \cdot 1! + 2 \cdot 2! + \ldots + n \cdot n! = (n+1)!-1$
  \end{enumerate}
  \hr
  \begin{enumerate}[a)]
    \item Funktioniert
    \item Funktioniert
    \item Funktioniert
    \item Funktioniert
    \item Funktioniert
    \item Funktioniert
  \end{enumerate}
\end{card}

\begin{card}
  Der reguläre Ausdruck \texttt{(10){n}} steht für eine Folge von binären Ziffern.Zeigen Sie, dass der Ausdruck für alle
  $n \geq 1$ dem Wert $\frac{2(4^n - 1)}{3}$ entspricht.
  \hr
  TODO
\end{card}

\begin{card}
  Jemand stellt die Behauptung auf $\forall n : n \in \N \rightarrow \varphi(n)$ mit $\varphi(n) = (n+2 = n+1)$ und beweist sie unter Hinweis auf $\varphi(n) \Rightarrow \varphi(n^+)$.
  \begin{enumerate}[a)]
    \item Zeigen Sie, dass der Induktionsschritt gültig ist.
    \item Zeigen Sie, dass die Verallgemeinerung, also der Induktionsschluss falsch ist.
  \end{enumerate}
  \hr
  TODO
\end{card}

\begin{card}
  Zeigen Sie mit Hilfe der vollständigen Induktion, dass die Anzahl der Elemente der Potenzmenge $P(A)$ einer endlichen
  Menge $A$ gleich $2^{|A|}$ ist.
  \hr
  TODO
\end{card}

\begin{card}
  Einer der Grundsätze der Zahlentheorie besagt, dass alle natürlichen Zahlen in Faktoren aus Primzahlen zerlegt werden
  können: $m = p_1^{e_1} \cdot \ldots \cdot p_n^{e_n}$. Beweisen Sie den Grundsatz mit Hilfe der Wertverlaufsinduktion.
  \hr
  TODO
\end{card}

\begin{card}
  In der Analysis gilt die folgende verallgemeinerte Produktregel:
  \[
  (f_1 \cdot f_2 \cdot \ldots \cdot f_n)' = (f_1' \cdot f_2 \cdot \ldots \cdot f_n) + (f_1 \cdot f_2' \cdot \ldots \cdot f_n) + \cdot + (f_1 \cdot f_2 \cdot \ldots \cdot f_n')
  \]
  für die differenzierbare Funktionen $f_i$. Beweisen Sie den Satz mit Hilfe der Wertverlaufsinduktion.
  \hr
  \begin{align*}
    \lrk f_1 \rrk ' &= f_1' \\
    (f_1 \cdot f_2 \cdot \ldots \cdot f_{n+1})' &= (f_1' \cdot \ldots \cdot f_{n+1}) + \ldots + (f_1 \cdot \ldots \cdot f_{n+1}') \\
    &= ((f_1 \cdot f_2 \cdot \ldots \cdot f_n)(f_{n+1}))' \\
    &= (f_1 \cdot f_2 \cdot \ldots \cdot f_n)' \cdot f_{n+1} + (f_1 \cdot f_2 \cdot \ldots \cdot f_n) \cdot (f_{n+1})' \\
    &= f_1' \cdot f_2 \cdot \ldots \cdot f_n \cdot f_{n+1} + \ldots \\
    & \hspace{3.3cm} + f_1 \cdot f_2 \cdot \ldots \cdot f_n' \cdot f_{n+1} \\
    & \hspace{3.3cm} + f_1 \cdot f_2 \cdot \ldots \cdot f_n \cdot f_{n+1}' \\
    &= (f_1 \cdot f_2 \cdot \ldots \cdot f_{n+1})' \\
  \end{align*}
\end{card}

\begin{card}
	Die strukturelle Induktion setzt sich die Idee der vollständigen Induktion auf Mengen um. Wie lautet entsprechend:
  \begin{enumerate}[a)]
	  \item die Verankerung
	  \item der Induktionsschritt
	  \item die Verallgemeinerung
	\end{enumerate}
	bei der strukturellen Induktion?
	\hr
	\begin{enumerate}
	  \item Beweisen für die Grundelemente
	  \item Zeigen, dass sich grüßere Elemente rekursiv aus kleineren Elementen zusammensetzen.
	  \item Ist es für die Grundelemente und den Induktionsschritt bewiesen, so gilt die Verallgemeinerung.
	\end{enumerate}
\end{card}

\begin{card}
  Die Folge der Fibonacci-Zahlen ist für $n \in \N \, (\geq 1)$ wie folgt definiert ($m := n+2$):
  \[
    a_{m+2} =
\begin{cases}
  a_{m+1} + a_{m} & \qquad m > 0 \\
  1 & \qquad \text{sonst}
\end{cases}
  \]
  Beweisen Sie mit Hilfe struktureller Induktion für ($a_n$) die Gültigkeit folgender Sätze:
  \begin{enumerate}[a)]
    \item $1 + a_1 + a_2 + \ldots + a_n = a_{n+2}$
    \item $3 \mid n \Rightarrow a_{n}$ ist gerade und $3 \nmid n \Rightarrow a_{n}$ ist ungerade
	\end{enumerate}
	\hr
  \begin{enumerate}[a)]
	  \item
	    \begin{align*}
        n = 1: 1+a_1 &= a_{1+2} \\
        2 &= 2 \\
        n = 2: 1+a_1+a_2 &= a_{2+2} \\
        3 &= 3 \\
        1 + a_1 + \ldots + a_n + a_{n+1} &= a_{(n+1)+2} \\
        a_{n+2} + a_{n+1} &= a_{n+3} \\
	    \end{align*}
	  \item TODO
	\end{enumerate}
\end{card}

\begin{card}
  Zeigen Sie mit Hilfe der doppelten Induktion, dass xor für boolsche Ausdrücke nicht universell ist. (Hinweis: Die
  Tautologie ist nicht dargestellbar.)
	\hr
  \begin{tabular}{cc|c|c|c|c|c}
    a & b & $\oplus$ & $\land$ & $\lor$ & $\top$ & $\bot$ \\ \hline
    0 & 0 & 0 & 0 & 0 & 1 & 0 \\
    0 & 1 & 1 & 0 & 1 & 1 & 0 \\
    1 & 0 & 1 & 0 & 1 & 1 & 0 \\
    1 & 1 & 0 & 1 & 1 & 1 & 0 \\
  \end{tabular}
  TODO
  % \begin{align*}
  %   a \oplus b \oplus a \oplus b \Rightalign a \oplus \ldots \oplus a \oplus a \oplus \ldots \oplus a \\
  % \end{align*}
\end{card}

\begin{card}
	Was bedeutet es, wenn die Verankerung bei $n=a$, also z.B. $n=5$ bewiesen wird, aber nicht für kleinere Werte?
	\hr
	Bewiesen erst ab $n=5$ und aufwärts, bzw. Voraussetzung erst ab dann beweisbar anwendbar.
\end{card}

\begin{card}
	Kann man aus der Allgemeingültigkeit von $\varphi$ schließen, dass $\varphi (0)$ und $\varphi (n)\Rightarrow \varphi (n^+)$ gelten?
	\hr
  Durch die Allgemeingültigkeit ist $\varphi(0)$, $\varphi(n)$ und $\varphi(n^+)$ gültig. Aus $T \Rightarrow T$ folgt auch wieder $T$. Somit ja.
\end{card}

\begin{card}
	Angenommen $\varphi$ wird für 0 bewiesen, ist für ein $n= a > 0$ ungültig und $\varphi(n) \Rightarrow \varphi(n+1)$ kann wiederum gezeigt werden. Was besagt das für die Induktion?
	\hr
	Durch den gültigen Nachfolger, muss es für alle $a > 0$ gelten. Da es für ein $a > 0$ nicht gilt, ist entweder der
	Schritt oder die Verankerung falsch.
\end{card}

\begin{card}
	Darf die bewiesene Verankerung im Induktionsschritt verwendet werden?
	\hr
	Der Induktionsschritt ist die Verallgemeinerung. Die Verankerung wird auf konkreten Werten angewendet. Daraus folgt: Nein, darf man nicht.
\end{card}

\begin{card}
	Das Schema der vollständigen Induktion lautet:\\
  \begin{minipage}[t]{0.48\textwidth}
    Gilt für eine Menge A:\\
    $0 \in A$\\
    $x \in A \Rightarrow x^+ \in A$\\
    $\vDash \mathbb{N} \subseteq A$\\
	\end{minipage}
  \begin{minipage}[t]{0.48\textwidth}
    Gilt für ein auf $\mathbb{N}$ definiertes Prädikat $\varphi$:\\
    $\varphi(0)$\\
    $\varphi(x) \Rightarrow \varphi(x^+)$\\
    $\vDash \forall x (\varphi(x))$\\
	\end{minipage}
	Geben Sie wenigstens 3 Möglichkeiten der Verallgemeinerung (Abstrahierung) an.
	\hr
	\begin{enumerate}[a)]
    \item beliebiges Startelement $s$, statt 0
    \item mehrere Startelemente / Verankerungen
    \item andere Nachfolgerfunktion
    \item $\N$ kann durch eine beliebige Menge ersetzt werden
    \item mehrstelliges Prädikat $\varphi$
	\end{enumerate}
\end{card}
