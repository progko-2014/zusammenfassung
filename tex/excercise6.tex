\begin{card}
	\frametitle{Übung 5: $\lambda$-Kalkül}
	\url{http://people.f4.htw-berlin.de/~hebold/htw/pka/exercises/algorithmen-lambdaCalculus.pdf}
\end{card}

\begin{card}
	Das $\lambda$-Kalkül unterscheidet zwei Arten von Ausdrücken: Auswertungen und Abstraktionen. Benennen Sie für jeden der Ausdrücke dessen Art und dann innerhalb des Ausdrucks gebundene Variablen und Rumpf bzw. Funktionsargument und Funktion. 
	\begin{enumerate}[a)]
	\item $\lambda a.(a \quad \lambda b.(b \quad a))$
	\item $\lambda x.\lambda y.\lambda z.((z \quad x) \quad (z \quad y))$
	\end{enumerate}
	\hr
	Auswertung: in Klammern, hat Argumente\\
	Abstraktion: hat \textit{keine} Argumente, $\sim$ Funktion
	Achtung: Nicht verwechseln mit Rumpf, der auch in Klammern stehen kann.
	\begin{enumerate}[a)]
	\item Abstraktion, da nicht in Klammern\\
	Funktion: $\lambda a.f$\\
	Rumpf: $(a \quad \lambda b.(b \quad a))$\\
	Gebundene Variablen: $a$ (was ist mit b?)
	\item wie 1)\\
	Funktion: $\lambda x.\lambda y.\lambda z.f$ \\
	Rumpf: $((z \ x) \quad (z \quad y))$\\
	Gebundene Variablen: $x,y,z$
	\end{enumerate}
\end{card}

\begin{card}
	Das $\lambda$-Kalkül unterscheidet zwei Arten von Variablen: gebundene und freie. Benennen Sie für jeden der folgenden Ausdrücke diese.
	\begin{enumerate}[a)]
	\item $\lambda x.\lambda y.(\lambda x.y \quad \lambda y.x)$
	\item $\lambda x.(x \quad (\lambda y.(\lambda x.x \quad y) \quad x))$
	\end{enumerate}
	\hr
	Alle Variablen sind gebunden.
\end{card}

\begin{card}
	Werten Sie folgende $\lambda$-Ausdrücke aus:
	\begin{enumerate}[a)]
	\item $((\lambda x.\lambda y.(y \quad x) \quad \lambda p.\lambda q.p) \quad \lambda i.i)$
	\item $(((\lambda x.\lambda y.\lambda z((x \quad y) \quad z) \quad \lambda f.\lambda a.(f \quad a)) \quad \lambda i.i) \quad\lambda j.j)$
	\end{enumerate}
	\hr
	Der \underline{Ausdruck} wird als Wert in dem/der \textbf{Symbol/Variable} ersetzt, festgelegt durch $\lambda\mathbf{Symbol}$. Beachte die Klammern um den jeweilige Funktion, die erst die Auswertung erlaubt.
	\begin{enumerate}[a)]
	\item
	$(\textbf{(}\lambda \mathbf{x}.\lambda y.(y \quad \mathbf{x}) \quad \uline{\lambda p.\lambda q.p}\textbf{)} \quad \lambda i.i) 
	\Rightarrow 
	(\lambda y.(y \quad \lambda p.\lambda q.p) \quad \lambda i.i)$\\
	
	$\textbf{(}\lambda \mathbf{y}.(\mathbf{y} \quad \lambda p.\lambda q.p) \quad \uline{\lambda i.i}\textbf{)}
	\Rightarrow
	(\lambda i.i \quad \lambda p.\lambda q.p)$\\
	
	$\textbf{(}\lambda \mathbf{i}.\mathbf{i} \quad \uline{\lambda p.\lambda q.p}\textbf{)}
	\Rightarrow
	\lambda p.\lambda q.p$
	\item 
	$((\textbf{(}\lambda \mathbf{x}.\lambda y.\lambda z((\mathbf{x} \quad y) \quad z) \quad \uline{\lambda f.\lambda a.(f \quad a)}\textbf{)} \quad \lambda i.i) \quad \lambda j.j) \Rightarrow
	((\lambda y.\lambda z((\lambda f.\lambda a.(f \quad a) \quad y) \quad z) \quad \lambda i.i) \quad \lambda j.j)$
	\vfill
	$(\textbf{(}\lambda \textbf{y}.\lambda z((\lambda f.\lambda a.(f \quad a) \quad \textbf{y}) \quad z) \quad \uline{\lambda i.i} \textbf{)} \quad \lambda j.j) \Rightarrow 
	(\lambda z((\lambda f.\lambda a.(f \quad a) \quad \lambda i.i) \quad z) \quad \lambda j.j)$
	\vfill
	$\textbf{(}\lambda \textbf{z}((\lambda f.\lambda a.(f \quad a) \quad \lambda i.i) \quad \textbf{z}) \quad \uline{\lambda j.j} \textbf{)} \Rightarrow 
	((\lambda f.\lambda a.(f \quad a) \quad \lambda i.i) \quad \lambda j.j)$
	\vfill
	$(\textbf{(}\lambda \textbf{f}.\lambda a.(\textbf{f} \quad a) \quad \uline{\lambda i.i}\textbf{)} \quad \lambda j.j) \Rightarrow
	(\lambda a.(\lambda i.i \quad a) \quad \lambda j.j)$
	
	$\textbf{(}\lambda \textbf{a}.(\lambda i.i \quad \textbf{a}) \quad \uline{\lambda j.j}\textbf{)} \Rightarrow
	(\lambda i.i \quad \lambda j.j)$
	
	$\textbf{(}\lambda \textbf{i}.\textbf{i} \quad \uline{\lambda j.j}\textbf{)} \Rightarrow
	\lambda j.j$
	\end{enumerate}
\end{card}

\begin{card}
	Gegeben sind folgende $\lambda$-Ausdrücke:
	\begin{itemize}
	\item def $id = \lambda x.x$
	\item def $apply = \lambda f. \lambda x.(f \quad x)$
	\end{itemize}
	Zeigen Sie, dass $id = (apply \quad (apply \quad id))$
	\hr
	Hinweis: Literale aus verschiedenen eingesetzten $\lambda$-Ausdrücken sind nicht identisch trotz gleichem Namens.
	$(apply \quad (apply \quad id)) \Rightarrow (apply \quad (\lambda f. \lambda x.(f \quad x) \quad id))$
	
	$(apply \quad \textbf{(}\lambda \textbf{f}. \lambda x.(\textbf{f} \quad x) \quad \uline{id}\textbf{)}) \Rightarrow 
	(apply \quad \lambda x.(id \quad x))$
	
	$(apply \quad \lambda x0.\textbf{(} \lambda \textbf{x1} . \textbf{x1} \quad \uline{x0} \textbf{)}) \Rightarrow 
	(apply \quad \lambda x0.x0)$
	
	$(apply \quad \lambda x0.x0) \Rightarrow (\lambda f. \lambda x2.(f \quad x2) \quad \lambda x0.x0)$
	
	$\textbf{(}\lambda \textbf{f}. \lambda x2.(\textbf{f} \quad x2) \quad \uline{\lambda x0.x0} \textbf{)} \Rightarrow
	\lambda x2.(\lambda x0.x0 \quad x2)$
	
	$\lambda x2.\textbf{(}\lambda \textbf{x0}.\textbf{x0} \quad \uline{x2} \textbf{)} \Rightarrow
	\lambda x2.x2 \Leftrightarrow \lambda x.x$
\end{card}

\begin{card}
	Gegeben sind folgende $\lambda$-Ausdrücke:
	\begin{itemize}
	\item def $apply = \lambda f. \lambda x.(f \quad x)$
	\item def $pair = \lambda x. \lambda y. \lambda z.((z \quad x ) \quad y)$
	\end{itemize}
	Zeigen Sie, dass $apply = \lambda x.\lambda y.(((pair \quad x) \quad y) \quad id)$
	\hr
\end{card}

\begin{card}
	Gegeben sind folgende $\lambda$-Ausdrücke:
	\begin{itemize}
	\item def $id = \lambda x.x$
	\item def $apply = \lambda f. \lambda x.(f \quad x)$
	\item def $pair = \lambda x. \lambda y. \lambda z.((z \quad x ) \quad y)$
	\item def $self = \lambda x.(x \quad x)$
	\item def $second = \lambda x.\lambda y. y$
	\end{itemize}
	Zeigen Sie, dass $id = (self \quad (self \quad second))$
	\hr
\end{card}

\begin{card}
	Wieso wird das abstrakteste $\lambda$-Kalkül als \textbf{typfrei} bezeichnet?
	\hr
	Arbeit nur auf Symbolen, reine Textersetzung
\end{card}

\begin{card}
	Bei $\lambda$-Kalkül–Ausdrücken wird von Auswertung und Abstraktion gesprochen. Erklären Sie an einem Beispiel, in wiefern bei $\lambda$-Kalkül–Ausdrücken abstrahiert und	konkretisiert wird.
	\hr
\end{card}

\begin{card}
	Mit welchen Programmierkonzept aus C ist das typfreie $\lambda$-Kalkül vergleichbar? 
	\begin{enumerate}[a)]
	\item Makros
	\item Templates
	\item Funktionen
	\item Pointern
	\end{enumerate}
	\hr
	Makros, da diese keine Textersetzung durchführen.
\end{card}

\begin{card}
	Im Zusammenhang mit der Auswertung von $\lambda$-Ausdrücken kann es zu Namenskonflikten	kommen, die mit Hilfe der sogenannten
	$\alpha$
	Konvertierung gelöst werden. Erklären Sie an einem Beispiel die Problematik. Wie wird das Problem konkret gelöst?
	\hr
  Die $\alpha$-Konvertierung ist das Umbenennen der Elemente. Bei $(\lambda x.(x \quad x) \quad \lambda x.(x \quad x))$ kommt es zu einem Konflikt. Nach
  der Konvertierung: $(\lambda x.(x \quad x) \quad \lambda f.(f \quad f))$
\end{card}

\begin{card}
	Die Auswertung von Ausdrücken wird als $\beta$–Reduktion bezeichnet. Welches Problem tritt hier auf? (Beispiel!)
	\hr
	Es können Konflikte auftreten, die durch die $\alpha$-Konvertierung gelöst werden müssen.

	Beispiel: $(\lambda x.(x \quad x) \quad \lambda x.(x \quad x))$
\end{card}

\begin{card}
	Die einzige Möglichkeit im $\lambda$-Kalkül eine Wiederholung zu formulieren basiert auf dem sogenannten Fixpunktsatz. Was ist damit gemeint? Wieso lösen Fixpunkte das Problem der Wiederholung? 
	\hr
	Tafelanachrieb?
\end{card}
