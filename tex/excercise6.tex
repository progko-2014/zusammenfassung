\begin{card}
  \frametitle{Übung 6: $\lambda$-Kalkül}
  \url{http://people.f4.htw-berlin.de/~hebold/htw/pka/exercises/algorithmen-lambdaCalculus.pdf}
\end{card}

\begin{card}
  Das $\lambda$-Kalkül unterscheidet zwei Arten von Ausdrücken: Auswertungen und Abstraktionen. Benennen Sie für jeden der Ausdrücke dessen Art und dann innerhalb des Ausdrucks gebundene Variablen und Rumpf bzw. Funktionsargument und Funktion.
  \begin{enumerate}[a)]
    \item $\lambda a.(a \quad \lambda b.(b \quad a))$
    \item $\lambda x.\lambda y.\lambda z.((z \quad x) \quad (z \quad y))$
  \end{enumerate}
  \hr
  Auswertung: in Klammern, hat Argumente\\
  Abstraktion: hat \textit{keine} Argumente, entspricht Funktion
  Achtung: Nicht verwechseln mit Rumpf, der auch in Klammern stehen kann.
  \begin{enumerate}[a)]
    \item Abstraktion, da nicht in Klammern\\
      Funktion: $\lambda a.f$\\
      Rumpf: $(a \quad \lambda b.(b \quad a))$\\
      Gebundene Variablen: $a, b$
    \item Abstraktion, da nicht in Klammern\\
      Funktion: $\lambda x.\lambda y.\lambda z.f$ \\
      Rumpf: $((z \ x) \quad (z \quad y))$\\
      Gebundene Variablen: $x,y,z$
  \end{enumerate}
\end{card}

\begin{card}
  Das $\lambda$-Kalkül unterscheidet zwei Arten von Variablen: gebundene und freie. Benennen Sie für jeden der folgenden Ausdrücke diese.
  \begin{enumerate}[a)]
    \item $\lambda x.\lambda y.(\lambda x.y \quad \lambda y.x)$
    \item $\lambda x.(x \quad (\lambda y.(\lambda x.x \quad y) \quad x))$
  \end{enumerate}
  \hr
  Alle Variablen sind gebunden.
\end{card}

\begin{card}
  Werten Sie folgende $\lambda$-Ausdrücke aus:
  \begin{enumerate}[a)]
    \item $((\lambda x.\lambda y.(y \quad x) \quad \lambda p.\lambda q.p) \quad \lambda i.i)$
    \item $(((\lambda x.\lambda y.\lambda z((x \quad y) \quad z) \quad \lambda f.\lambda a.(f \quad a)) \quad \lambda i.i) \quad\lambda j.j)$
  \end{enumerate}
  \hr
  Der \underline{Ausdruck} wird als Wert in dem/der \textbf{Symbol/Variable} ersetzt, festgelegt durch $\lambda\mathbf{Symbol}$. Beachte die Klammern um den jeweilige Funktion, die erst die Auswertung erlaubt.
  \begin{enumerate}[a)]
    \item
      $(\textbf{(}\lambda \mathbf{x}.\lambda y.(y \quad \mathbf{x}) \quad \uline{\lambda p.\lambda q.p}\textbf{)} \quad \lambda i.i)
      \Rightarrow
      (\lambda y.(y \quad \lambda p.\lambda q.p) \quad \lambda i.i)$\\

      $\textbf{(}\lambda \mathbf{y}.(\mathbf{y} \quad \lambda p.\lambda q.p) \quad \uline{\lambda i.i}\textbf{)}
      \Rightarrow
      (\lambda i.i \quad \lambda p.\lambda q.p)$\\

      $\textbf{(}\lambda \mathbf{i}.\mathbf{i} \quad \uline{\lambda p.\lambda q.p}\textbf{)}
      \Rightarrow
      \lambda p.\lambda q.p$
    \item
      $((\textbf{(}\lambda \mathbf{x}.\lambda y.\lambda z.((\mathbf{x} \quad y) \quad z) \quad \uline{\lambda f.\lambda a.(f \quad a)}\textbf{)} \quad \lambda i.i) \quad \lambda j.j) \Rightarrow
      ((\lambda y.\lambda z.((\lambda f.\lambda a.(f \quad a) \quad y) \quad z) \quad \lambda i.i) \quad \lambda j.j)$
      \vfill
      $(\textbf{(}\lambda \textbf{y}.\lambda z.((\lambda f.\lambda a.(f \quad a) \quad \textbf{y}) \quad z) \quad \uline{\lambda i.i} \textbf{)} \quad \lambda j.j) \Rightarrow
      (\lambda z((\lambda f.\lambda a.(f \quad a) \quad \lambda i.i) \quad z) \quad \lambda j.j)$
      \vfill
      $\textbf{(}\lambda \textbf{z}.((\lambda f.\lambda a.(f \quad a) \quad \lambda i.i) \quad \textbf{z}) \quad \uline{\lambda j.j} \textbf{)} \Rightarrow
      ((\lambda f.\lambda a.(f \quad a) \quad \lambda i.i) \quad \lambda j.j)$
      \vfill
      $(\textbf{(}\lambda \textbf{f}.\lambda a.(\textbf{f} \quad a) \quad \uline{\lambda i.i}\textbf{)} \quad \lambda j.j) \Rightarrow
      (\lambda a.(\lambda i.i \quad a) \quad \lambda j.j)$

      $\textbf{(}\lambda \textbf{a}.(\lambda i.i \quad \textbf{a}) \quad \uline{\lambda j.j}\textbf{)} \Rightarrow
      (\lambda i.i \quad \lambda j.j)$

      $\textbf{(}\lambda \textbf{i}.\textbf{i} \quad \uline{\lambda j.j}\textbf{)} \Rightarrow
      \lambda j.j$
  \end{enumerate}
\end{card}

\begin{card}
  Gegeben sind folgende $\lambda$-Ausdrücke:
  \begin{itemize}
  \item def $id = \lambda x.x$
  \item def $apply = \lambda f. \lambda x.(f \quad x)$
  \end{itemize}
  Zeigen Sie, dass $id = (apply \quad (apply \quad id))$
  \hr
  Hinweis: Literale aus verschiedenen eingesetzten $\lambda$-Ausdrücken sind nicht identisch trotz gleichen Namens.
  \vfill
  $(apply \quad (\textbf{apply} \quad id)) \Rightarrow (apply \quad (\lambda f. \lambda x.(f \quad x) \quad id))$
  \vfill
  $(apply \quad \textbf{(}\lambda \textbf{f}. \lambda x.(\textbf{f} \quad x) \quad \uline{id}\textbf{)}) \Rightarrow
  (apply \quad \lambda x.(id \quad x))$
  \vfill
  $(apply \quad \lambda x_{0}.\textbf{(} \lambda \textbf{$x_{1}$} . \textbf{$x_{1}$} \quad \uline{x_{0}} \textbf{)}) \Rightarrow
  (apply \quad \lambda x_{0}.x_{0})$
  \vfill
  $(\textbf{apply} \quad \lambda x_{0}.x_{0}) \Rightarrow (\lambda f. \lambda x_{2}.(f \quad x_{2}) \quad \lambda x_{0}.x_{0})$
  \vfill
  $\textbf{(}\lambda \textbf{f}. \lambda x_{2}.(\textbf{f} \quad x_{2}) \quad \uline{\lambda x_{0}.x_{0}} \textbf{)} \Rightarrow
  \lambda x_{2}.(\lambda x_{0}.x_{0} \quad x_{2})$
  \vfill
  $\lambda x_{2}.\textbf{(}\lambda \textbf{$x_{0}$}.\textbf{$x_{0}$} \quad \uline{x_{2}} \textbf{)} \Rightarrow
  \lambda x_{2}.x_{2} \Leftrightarrow \lambda x.x$
\end{card}

\begin{card}
  Gegeben sind folgende $\lambda$-Ausdrücke:
  \begin{itemize}
    \item def $apply = \lambda f. \lambda x.(f \quad x)$
    \item def $id = \lambda x.x$
    \item def $pair = \lambda x. \lambda y. \lambda z.((z \quad x ) \quad y)$
  \end{itemize}
  Zeigen Sie, dass $apply = \lambda x.\lambda y.(((pair \quad x) \quad y) \quad id)$
  \hr
  $\lambda x_{0}.\lambda y_{0}.(((\textbf{pair} \quad x_{0}) \quad y_{0}) \quad id) \Rightarrow 
  \lambda x_{0}.\lambda y_{0}.(((\lambda x_{1}. \lambda y_{1}. \lambda z.((z \quad x_{1} ) \quad y_{1}) \quad x_{0}) \quad y_{0}) \quad id)$
  \vfill
  $\lambda x_{0}.\lambda y_{0}.((\textbf{(}\lambda \mathbf{x_{1}}. \lambda y_{1}. \lambda z.((z \quad \mathbf{x_{1}} ) \quad y{1}) \quad \uline{x_{0}} \textbf{)} \quad y_{0}) \quad id) \Rightarrow 
  \lambda x_{0}.\lambda y_{0}.((\lambda y_{1}. \lambda z.((z \quad x_{0} ) \quad y_{1}) \quad y_{0}) \quad id)$
  \vfill
  $\lambda x_{0}.\lambda y_{0}.(\textbf{(} \lambda \mathbf{y_{1}}. \lambda z.((z \quad x_{0} ) \quad \mathbf{y_{1}}) \quad \uline{y_{0}} \textbf{)} \quad id) \Rightarrow
  \lambda x_{0}.\lambda y_{0}.( \lambda z.((z \quad x_{0} ) \quad y_{0}) \quad id)$
  \vfill
  $\lambda x_{0}.\lambda y_{0}.\textbf{(} \lambda \textbf{z}.((\textbf{z} \quad x_{0} ) \quad y_{0})\quad \uline{id} \textbf{)}  \Rightarrow
  \lambda x_{0}.\lambda y_{0}.((id \quad x_{0} ) \quad y_{0})$
  \vfill
  $\lambda x_{0}.\lambda y_{0}.((\textbf{id} \quad x_{0} ) \quad y_{0}) \Rightarrow 
  \lambda x_{0}.\lambda y_{0}.((\lambda x_{2}.x_{2} \quad x_{0} ) \quad y_{0})$
  \vfill
  $\lambda x_{0}.\lambda y_{0}.(\textbf{(}\lambda \mathbf{x_{2}}.\mathbf{x_{2}} \quad \uline{x_{0}} \textbf{)} \quad y_{0}) \Rightarrow 
  \lambda x_{0}.\lambda y_{0}.(x_{0} \quad y_{0}) \Leftrightarrow \lambda f. \lambda x.(f \quad x)$
\end{card}

\begin{card}
  Gegeben sind folgende $\lambda$-Ausdrücke:
  \begin{itemize}
    \item def $id = \lambda x.x$
    \item def $self = \lambda x.(x \quad x)$
    \item def $second = \lambda x.\lambda y. y$
  \end{itemize}
  Zeigen Sie, dass $id = (self \quad (self \quad second))$
  \hr
  $(self \quad (\textbf{self} \quad second)) \Rightarrow (self \quad (\lambda x.(x \quad x) \quad second))$
  \vfill
  $(self \quad \textbf{(}\lambda \textbf{x}.(\textbf{x} \quad \textbf{x}) \quad \uline{second}\textbf{)}) \Rightarrow 
  (self \quad (second \quad second))$
  \vfill
  $(\textbf{self} \quad (second \quad second)) \Rightarrow 
  (\lambda x.(x \quad x) \quad (second \quad second))$
  \vfill
  $\textbf{(}\lambda \textbf{x}.(\textbf{x} \quad \textbf{x}) \quad \uline{(second \quad second)}\textbf{)} \Rightarrow
  ((second \quad second) \quad (second \quad second))$
  \vfill
  $((\textbf{second} \quad second) \quad (second \quad second)) \Rightarrow ((\lambda x.\lambda y. y \quad second) \quad (second \quad second))$
  \vfill
  $(\textbf{(}\lambda \textbf{x}.\lambda y. y \quad \uline{second}\textbf{)} \quad (second \quad second)) \Rightarrow (\lambda y. y \quad (second \quad second))$
  \vfill
  $\textbf{(}\lambda \textbf{y}. \textbf{y} \quad \uline{(second \quad second)}\textbf{)} \Rightarrow (second \quad second)$
  \vfill
  $(\textbf{second} \quad second) \Rightarrow (\lambda x.\lambda y. y \quad second)$
  \vfill
  $\textbf{(}\lambda \textbf{x}.\lambda y. y \quad \uline{second}\textbf{)} \Rightarrow \lambda y. y \Leftrightarrow \lambda x.x$
\end{card}

\begin{card}
	Definieren mit Hilfe des $\lambda$-Kalküls die:
	\begin{enumerate}[a)]
    \item booleschen Werte true und false
    \item die Implikation
    \item die Äquivalenz
	\end{enumerate}
	\hr
	\begin{enumerate}[a)]
    \item $\top \equiv \lambda x.\lambda y . x$\\
        $\bot \equiv \lambda x.\lambda y . y$
    \item $if \equiv \lambda c.\lambda t.\lambda e.((c \quad t) \quad e) \quad \ent \quad c \; ? \; t : e$ \\
        $impl \equiv \lambda a.\lambda b.((a \quad b) \quad \top) \quad \ent \quad a \rightarrow b = a \; ? \; b : \top$ \\
        Probieren über Binden von: $((impl \quad \bot) \quad \bot) = \top$ ...
    \item	$not \equiv \lambda z.((z \quad \bot) \quad \top)$\\
        $equiv \equiv  \lambda a.\lambda b.((a \quad b) \quad (not \quad b)) \quad \ent \quad a \Leftrightarrow b = a \; ? \; b : \neg b$
	\end{enumerate}
\end{card}

\begin{card}
	Wieso wird das abstrakteste $\lambda$-Kalkül als \textbf{typfrei} bezeichnet?
	\hr
	Arbeit nur auf Symbolen, reine Textersetzung
\end{card}

\begin{card}
  Bei $\lambda$-Kalkül--Ausdrücken wird von Auswertung und Abstraktion gesprochen. Erklären Sie an einem Beispiel, in wiefern bei $\lambda$-Kalkül--Ausdrücken abstrahiert und konkretisiert wird.
  \hr
  Abtraktion: Funktion $\lambda x.x$\\
  Konkretisierung: Auswertung einer Funktion $(\lambda x.x \quad a)$
\end{card}

\begin{card}
  Mit welchen Programmierkonzept aus C ist das typfreie $\lambda$-Kalkül vergleichbar?
  \begin{enumerate}[a)]
    \item Makros
    \item Templates
    \item Funktionen
    \item Pointern
  \end{enumerate}
  \hr
  Makros, da diese nur Textersetzung durchführen.
  \begin{enumerate}[a)]
    \item Ja, da Makros nur Textersetzung durchführen.
    \item Nein, da diese generische Datentypen sind und daher nicht typfrei.
    \item Nein, weil Funktionen nicht typfrei sind.
    \item (Ja, da Pointer nur auf Speicheradressen zeigen und typfrei sind?)
  \end{enumerate}
\end{card}

\begin{card}
  Im Zusammenhang mit der Auswertung von $\lambda$-Ausdrücken kann es zu Namenskonflikten kommen, die mit Hilfe der sogenannten
  $\alpha$--Konvertierung gelöst werden.
  \begin{enumerate}[a)]
    \item Erklären Sie an einem Beispiel die Problematik.
    \item Wie wird das Problem konkret gelöst?
  \end{enumerate}
  \hr
  \begin{enumerate}[a)]
    \item Die $\alpha$--Konvertierung ist das Umbenennen der Elemente. Bei $(\lambda x.(x \quad x) \quad \lambda x.(x \quad x))$ kommt es zu einem Konflikt.
    \item Vor der $\beta$--Reduktion muss eine Umbennenung durchgeführt werden. Nach der Konvertierung: $(\lambda x.(x \quad x) \quad \lambda f.(f \quad f))$
  \end{enumerate}
\end{card}

\begin{card}
  Die Auswertung von Ausdrücken wird als $\beta$–Reduktion bezeichnet. Welches Problem tritt hier auf? (Beispiel!)
  \hr
  \begin{itemize}
    \item Es können Namenskonflikte auftreten, die durch die $\alpha$-Konvertierung gelöst werden müssen.\\
      Beispiel: $(\lambda x.(x \quad x) \quad \lambda x.(x \quad x))$
    \item Endlosrekursion
  \end{itemize}
\end{card}

\begin{card}
  Die einzige Möglichkeit im $\lambda$-Kalkül eine Wiederholung zu formulieren basiert auf dem sogenannten Fixpunktsatz. Was ist damit gemeint? Wieso lösen Fixpunkte das Problem der Wiederholung?
  \hr
  Der Fixpunktsatz berechnet den Punkt, an dem die Eingabe auch die Ausgabe ist. Dieser ist definiert als
  \[
    \forall F \exists H ( \quad (F \quad H) = H \quad )
  \]
\end{card}
