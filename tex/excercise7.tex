\begin{card}
	\frametitle{Übung 7: $\lambda$-Kalkül - Fortsetzung}
	\url{http://people.f4.htw-berlin.de/~hebold/htw/pka/exercises/algorithmen-lambdaCalculus_cont.pdf}
\end{card}

\begin{card}
	Definieren Sie im $\lambda$–Kalkül die
	\begin{enumerate}[a)]
    \item die Null (=0)
    \item die Nachfolgefunktion succ
    \item eine beliebige Zahl
	\end{enumerate}
	\hr
	\begin{enumerate}[a)]
    \item $\lambdanumber{0} = \lambda x.x$
    \item $\lambdanumber{n} = [\bot, \lambdanumber{n}] = \lambda x.((x \quad \bot) \quad \lambdanumber{n})$
    \item $\lambdanumber{1} = [\bot, \lambdanumber{0}] = \lambda x.((x \quad \bot) \quad \lambdanumber{0})$\\
          $\lambdanumber{2} = [\bot, \lambdanumber{1}] = \lambda x.((x \quad \bot) \quad \lambdanumber{1})$\\
        \dots
	\end{enumerate}
\end{card}

\begin{card}
	Definieren Sie im $\lambda$–Kalkül die primitiv rekursiven Funktionen:
	\begin{enumerate}[a)]
    \item $0$
    \item $N$
    \item $P^m_i$
    \item $S^{n+1}$
    \item $R$
	\end{enumerate}
	\hr
	\begin{enumerate}[a)]
    \item $0 = \lambda x.\lambdanumber{0}$
    \item $(N \quad \lambdanumber{n}) = (\lambda y.[\bot, y] = \lambda y.\lambda x.((x \quad \bot) \quad y) \quad \lambdanumber{n})$\\
    \item $P^m_i = \lambda x_1,...,x_m.x_i$
    \item $S^{n+1} = \lambda x.(F \quad (G_1 \quad x) \quad \ldots \quad (G_n \quad x))$
    \item
      \begin{align*}
        (H \quad m^+) &= ((G \quad (H \quad m)) \quad m) \\
        ((H \quad m^+) \quad x) &= (((G \quad ((H \quad m)) \quad x) \quad m) \quad x)
      \end{align*}
	\end{enumerate}
\end{card}

\begin{card}
	Wie lautet der Fixpunkt von:
	\begin{enumerate}[a)]
    \item not
    \item succ
	\end{enumerate}
  \[ Y &= \lambda f.(\lambda x.(f(x \quad x) \quad \lambda x.(f(x \quad x))) \]
	\hr
	\begin{enumerate}[a)]
    \item
      \begin{align*}
        (Y \quad not) &= (\lambda f.(\lambda x.(f(x \quad x) \quad \lambda x.(f(x \quad x))) \quad not) \\
        &= (\lambda x.(not(x \quad x) \quad \lambda x.(not(x \quad x))) \\
        &= not(\lambda x.(not(x \quad x)) \quad \lambda x.(not(x \quad x))) \\
        &= not(Y \quad not) \\
      \end{align*}
    \item
      \begin{align*}
        (Y \quad succ) &= (\lambda f.(\lambda x.(f(x \quad x) \quad \lambda x.(f(x \quad x))) \quad succ) \\
        &= (\lambda x.(succ(x \quad x) \quad \lambda x.(succ(x \quad x))) \\
        &= succ(\lambda x.(succ(x \quad x)) \quad \lambda x.(succ(x \quad x))) \\
        &= succ(Y \quad succ) \\
      \end{align*}
	\end{enumerate}
\end{card}

\begin{card}
	Beschreiben Sie den Unterschied von $=$ (Gleichheit) und $\equiv$ (Identität) an einem Beispiel.
	\hr
	Gleichheit: Äquivalenz; kann auch Behauptung sein und soll sich logisch ergeben.\\
	Identität: Definition linker Seite durch rechte Seite, vgl. $A := B, A=_{def} B$ oder hier $A \equiv B $ für Definition von B zu A.
\end{card}

\begin{card}
	Bei der Definition der Substitution werden Funktionen auf zwei Arten miteinander verknüpft.
  \begin{enumerate}[a)]
	  \item Nennen Sie die beiden Arten.
	  \item Beschreiben Sie den Unterschied an einem Beispiel.
	\end{enumerate}
	\hr
  \begin{enumerate}[a)]
	  \item Komposition, Mehrstelligkeit (Currying)
    \item Komposition: $(f_1( \dots (f_{n-1}(f_n(a))) \dots)) = (f_1 \circ \ldots \circ f_{n-1} \circ f_n)(a)$\\
	  		Currying: $f(a_1)(a_2) \ldots (a_n) = f(a_1,a_2, \dots, a_n)$
	\end{enumerate}
\end{card}
