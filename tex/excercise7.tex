\begin{card}
	\frametitle{Übung 7: $\lambda$-Kalkül - Fortsetzung}
	\url{http://people.f4.htw-berlin.de/~hebold/htw/pka/exercises/algorithmen-lambdaCalculus_cont.pdf}
\end{card}

\begin{card}
	Definieren Sie im $\lambda$–Kalkül die
	\begin{enumerate}[a)]
    \item die Null (=0)
    \item die Nachfolgefunktion succ
    \item eine beliebige Zahl
	\end{enumerate}
	\hr
	\begin{enumerate}[a)]
    \item $^\lceil 0 ^\rceil = \lambda x.x$
    \item $^\lceil n+1 ^\rceil = [\bot, ^\lceil n ^\rceil] = \lambda x.((x \quad \bot) \quad ^\lceil n ^\rceil)$
    \item $^\lceil 1 ^\rceil = [\bot, ^\lceil 0 ^\rceil] = \lambda x.((x \quad \bot) \quad ^\lceil 0 ^\rceil)$\\
        $^\lceil 2 ^\rceil = [\bot, ^\lceil 1 ^\rceil] = \lambda x.((x \quad \bot) \quad ^\lceil 1 ^\rceil)$\\
        \dots
	\end{enumerate}
\end{card}

\begin{card}
	Definieren Sie im $\lambda$–Kalkül die primitiv rekursiven Funktionen:
	\begin{enumerate}[a)]
    \item $0$
    \item $N$
    \item $P^m_i$
    \item $S^{n+1}$
    \item $R$
	\end{enumerate}
	\hr
	\begin{enumerate}[a)]
    \item $0 = \lambda x.^\lceil 0 ^\rceil$
    \item $^\lceil n+1 ^\rceil = [\bot, ^\lceil n ^\rceil] = \lambda x.((x \quad \bot) \quad ^\lceil n ^\rceil)$\\
    \item $P^m_i = \lambda x_1,...,x_m.x_i$
    \item $S^{n+1} = \lambda x.(F \quad (G_1 \quad x) \quad \ldots \quad (G_n \quad x))$
    \item
      \begin{align*}
        (H \quad m^+) &= ((G \quad (H \quad m)) \quad m) \\
        ((H \quad m^+) \quad x) &= (((G \quad (H \quad m) \quad x) \quad m) \quad x)
      \end{align*}
	\end{enumerate}
\end{card}

\begin{card}
	Wie lautet der Fixpunkt von:
	\begin{enumerate}[a)]
    \item not
    \item succ
	\end{enumerate}
	\hr
	\begin{enumerate}[a)]
    \item
      \begin{align*}
        Y &= \lambda f.(\lambda x.(f(x \quad x) \quad \lambda x.(f(x \quad x))) \\
        (Y \quad not) &= (\lambda f.(\lambda x.(f(x \quad x) \quad \lambda x.(f(x \quad x))) \quad not) \\
        &= (\lambda x.(not(x \quad x) \quad \lambda x.(not(x \quad x))) \\
        &= not(\lambda x.(not(x \quad x)) \quad \lambda x.(not(x \quad x))) \\
        &= not(not(\lambda x.(not(x \quad x)) \quad \lambda x.(not(x \quad x)))) \\
        &= ... \text{ (Endlos-Auflösen)} \\
      \end{align*}
    \item $(Y \quad succ) = ...$ (Endlos-Auflösen)
	\end{enumerate}
\end{card}

\begin{card}
	Beschreiben Sie den Unterschied von $=$ (Gleichheit) und $\equiv$ (Identität) an einem Beispiel.
	\hr
	Gleichheit: Äquivalenz; kann auch Behauptung sein und soll sich logisch ergeben.\\
	Identität: Definition linker Seite durch rechte Seite, vgl. $A := B, A=_{def} B$ oder hier $A \equiv B $ für Definition von B zu A.
\end{card}

\begin{card}
	Bei der Definition der Substitution werden Funktionen auf zwei Arten miteinander verknüpft.
  \begin{enumerate}[a)]
	  \item Nennen Sie die beiden Arten.
	  \item Beschreiben Sie den Unterschied an einem Beispiel.
	\end{enumerate}
	\hr
  \begin{enumerate}[a)]
	  \item Komposition, Mehrstelligkeit (Currying)
	  \item Currying = $(\dots (f(a_1)a_2) \dots a_n) = f(a_1,a_2, \dots, a_n)$
	\end{enumerate}
\end{card}
