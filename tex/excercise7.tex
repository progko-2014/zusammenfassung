\begin{card}
	\frametitle{Übung 7: $\lambda$-Kalkül - Fortsetzung}
	\url{http://people.f4.htw-berlin.de/~hebold/htw/pka/exercises/algorithmen-lambdaCalculus_cont.pdf}
\end{card}

\begin{card}
	Definieren Sie im $\lambda$–Kalkül die
	\begin{enumerate}[a)]
	\item die Null (=0)
	\item die Nachfolgefunktion succ
	\item eine beliebige Zahl
	\end{enumerate}
	\hr
	\begin{enumerate}[a)]
	\item $^\lceil 0 ^\rceil = \lambda x.x$
	\item $^\lceil n+1 ^\rceil = \lambda x.((x \perp )\qquad ^\lceil n ^\rceil)$
	\item $^\lceil n ^\rceil = $beliebig häufig anwenden von 2.) ?
	\end{enumerate}
\end{card}

\begin{card}
	Definieren Sie im $\lambda$–Kalkül die primitiv rekursiven Funktionen: 
	\begin{enumerate}[a)]
	\item $0$
	\item $N$
	\item $P^m_n$
	\item $S^{n+1}$
	\end{enumerate}
\end{card}

\begin{card}
	Wie lautet der Fixpunkt von: 
	\begin{enumerate}[a)]
	\item not
	\item succ
	\end{enumerate}
	\hr
	Info: \url{https://de.wikipedia.org/wiki/Fixpunkt-Kombinator}
\end{card}

\begin{card}
	Definieren Sie im $\lambda$-Kalkül das primitiv rekursive Funktionsschema	\textbf{R}
\end{card}

\begin{card}
	Beschreiben Sie den Unterschied von $=$ (Gleichheit) und $\equiv$ (Identität) an einem Beispiel.
	\hr
	Gleichheit: Äquivalenz; kann auch Behauptung sein und soll sich logisch ergeben.\\
	Identität: Definition linker Seite durch rechte Seite, vgl. $A := B, A=_{def} B$ oder hier $A \equiv B $ für Definition von B zu A.
\end{card}

\begin{card}
	Bei der Definition der Substitution werden Funktionen auf zwei Arten miteinander verknüpft. Nennen Sie die beiden Arten. Beschreiben Sie den Unterschied an einem Beispiel.
	\hr
	Komposition, Mehrstelligkeit (Currying)
	Currying = $(\dots (f(a_1)a_2) \dots a_n) = f(a_1,a_2, \dots, a_n)$
\end{card}

\begin{card}
	Funktionen können durch Komposition oder Currying verknüpft werden.
	\begin{enumerate}[a)]
	\item Beschreiben Sie den Unterschied an einem Beispiel.
	\item Welche der beiden Verknüpfungsarten entspricht einer Funktionsdefinition in Java?
	\end{enumerate}
	\hr
	Java: immer mehrstellige Funktionen, Gegenteil von funktionalen Sprachen
\end{card}