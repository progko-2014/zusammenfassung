\begin{card}
	\frametitle{Übung 7: $\lambda$-Kalkül - Fortsetzung}
	\url{http://people.f4.htw-berlin.de/~hebold/htw/pka/exercises/algorithmen-lambdaCalculus_cont.pdf}
\end{card}

\begin{card}
	Definieren Sie im $\lambda$–Kalkül die
	\begin{enumerate}[a)]
    \item die Null (=0)
    \item die Nachfolgefunktion succ
    \item eine beliebige Zahl
	\end{enumerate}
	\hr
	\begin{enumerate}[a)]
    \item $^\lceil 0 ^\rceil = \lambda x.x$
    \item $^\lceil n+1 ^\rceil = [\bot, ^\lceil n ^\rceil] = \lambda x.((x \quad \bot) ^\lceil n ^\rceil)$
    \item $^\lceil 1 ^\rceil = [\bot, ^\lceil 0 ^\rceil] = \lambda x.((x \quad \bot) ^\lceil 0 ^\rceil)$\\
        $^\lceil 2 ^\rceil = [\bot, ^\lceil 1 ^\rceil] = \lambda x.((x \quad \bot) ^\lceil 1 ^\rceil)$
	\end{enumerate}
\end{card}

\begin{card}
	Definieren Sie im $\lambda$–Kalkül die primitiv rekursiven Funktionen:
	\begin{enumerate}[a)]
    \item $0$
    \item $N$
    \item $P^m_i$
    \item $S^{n+1}$
	\end{enumerate}
	\hr
	\begin{enumerate}[a)]
    \item $0 = \lambda x.^\lceil 0 ^\rceil$
    \item $^\lceil n+1 ^\rceil = [\bot, ^\lceil n ^\rceil] = \lambda x.((x \quad \bot) ^\lceil n ^\rceil)$\\
    \item $P^m_i = \lambda x_1,...,x_m.x_i$
    \item $S^{n+1} = \lambda x.(...((F (G_1 \quad x)) (G_2 \quad x)) ... (G_n \quad x))$
	\end{enumerate}
\end{card}

\begin{card}
	Wie lautet der Fixpunkt von:
	\begin{enumerate}[a)]
    \item not
    \item succ
	\end{enumerate}
	\hr
  \[Y = \lambda f.(\lambda x.(f(x \quad x) \lambda x.(f(x \quad x)))\]
	\begin{enumerate}[a)]
    \item $(Y \quad not) = ... (Endlos-Auflösen)$
    \item $(Y \quad succ) = ... (Endlos-Auflösen)$
	\end{enumerate}
	Info: \url{https://de.wikipedia.org/wiki/Fixpunkt-Kombinator}
\end{card}

\begin{card}
	Definieren Sie im $\lambda$-Kalkül das primitiv rekursive Funktionsschema \textbf{R}
	\hr
  $(H \quad m) = ((G \quad (H \quad m^-)) \quad m)$\\
  $\Leftrightarrow ((H \quad m^+) \quad x) = (((G \quad (H \quad m) \quad x) \quad m^+) \quad x)$
\end{card}

\begin{card}
	Beschreiben Sie den Unterschied von $=$ (Gleichheit) und $\equiv$ (Identität) an einem Beispiel.
	\hr
	Gleichheit: Äquivalenz; kann auch Behauptung sein und soll sich logisch ergeben.\\
	Identität: Definition linker Seite durch rechte Seite, vgl. $A := B, A=_{def} B$ oder hier $A \equiv B $ für Definition von B zu A.
\end{card}

\begin{card}
	Bei der Definition der Substitution werden Funktionen auf zwei Arten miteinander verknüpft.
  \begin{enumerate}[a)]
	  \item Nennen Sie die beiden Arten.
	  \item Beschreiben Sie den Unterschied an einem Beispiel.
	\end{enumerate}
	\hr
  \begin{enumerate}[a)]
	  \item Komposition, Mehrstelligkeit (Currying)
	  \item Currying = $(\dots (f(a_1)a_2) \dots a_n) = f(a_1,a_2, \dots, a_n)$
	\end{enumerate}
\end{card}
