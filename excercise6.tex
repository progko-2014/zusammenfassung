\begin{card}
	\frametitle{Übung 5: $\lambda$-Kalkül}
	\url{http://people.f4.htw-berlin.de/~hebold/htw/pka/exercises/algorithmen-lambdaCalculus.pdf}
\end{card}

\begin{card}
	Wieso wird das abstrakteste $\lambda$-Kalkül als \textbf{typfrei} bezeichnet?
	\hr
	Arbeit nur auf Symbolen, reine Textersetzung
\end{card}

\begin{card}
	Bei $\lambda$-Kalkül–Ausdrücken wird von Auswertung und Abstraktion gesprochen. Erklären Sie an einem Beispiel, in wiefern bei $\lambda$-Kalkül–Ausdrücken abstrahiert und	konkretisiert wird.
	\hr
	
\end{card}

\begin{card}
	Mit welchen Programmierkonzept aus C ist das typfreie $\lambda$-Kalkül vergleichbar? 
	\begin{enumerate}
	\item Makros
	\item Templates
	\item Funktionen
	\item Pointern
	\end{enumerate}
	\hr
\end{card}

\begin{card}
	Im Zusammenhang mit der Auswertung von $\lambda$-Ausdrücken kann es zu Namenskonflikten	kommen, die mit Hilfe der sogenannten
	$\alpha$
	Konvertierung gelöst werden. Erklären Sie an einem Beispiel die Problematik. Wie wird das Problem konkret gelöst?
	\hr
\end{card}

\begin{card}
	Die Auswertung von Ausdrücken wird als $\beta$–Reduktion bezeichnet. Welches Problem tritt hier auf? (Beispiel!)
	\hr
\end{card}

\begin{card}
	Die einzige Möglichkeit im $\lambda$-Kalkül eine Wiederholung zu formulieren basiert auf dem sogenannten Fixpunktsatz. Was ist damit gemeint? Wieso lösen Fixpunkte das Problem der Wiederholung? 
	\hr
	Tafelanachrieb?
\end{card}