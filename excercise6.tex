\begin{card}
	\frametitle{Übung 5: $\lambda$-Kalkül}
	\url{http://people.f4.htw-berlin.de/~hebold/htw/pka/exercises/algorithmen-lambdaCalculus.pdf}
\end{card}

\begin{card}
	Das $\lambda$-Kalkül unterscheidet zwei Arten von Ausdrücken: Auswertungen und Abstraktionen. Benennen Sie für jeden der Ausdrücke dessen Art und dann innerhalb des Ausdrucks gebundene Variablen und Rumpf bzw. Funktionsargument und Funktion. 
	\begin{enumerate}[a)]
	\item $\lambda a.(a \quad \lambda b.(b \quad a))$
	\item $\lambda x.\lambda y.\lambda z.((z \ x) \quad (z \quad y))$
	\end{enumerate}
	\hr
	Auswertung: in Klammern, hat Argumente\\
	Abstraktion: hat \textit{keine} Argumente, $\sim$ Funktion
	Achtung: Nicht verwechseln mit Rumpf, der auch in Klammern stehen kann.
	\begin{enumerate}[a)]
	\item Abstraktion, da nicht in Klammern\\
	Funktion: $\lambda a.f$\\
	Rumpf: $(a \quad \lambda b.(b \quad a))$\\
	Gebundene Variablen: $a$ (was ist mit b?)
	\item wie 1)\\
	Funktion: $\lambda x.\lambda y.\lambda z.f$ \\
	Rumpf: $((z \ x) \quad (z \quad y))$\\
	Gebundene Variablen: $x,y,z$
	\end{enumerate}
\end{card}


\begin{card}
	Das $\lambda$-Kalkül unterscheidet zwei Arten von Variablen: gebundene und freie. Benennen Sie für jeden der folgenden Ausdrücke diese.
	\begin{enumerate}[a)]
	\item $\lambda x.\lambda y.(\lambda x.y \quad \lambda y.x)$
	\item $\lambda x.(x \quad (\lambda y.(\lambda x.x \quad y) \quad x))$
	\end{enumerate}
	\hr
	Alle Variablen sind gebunden.
\end{card}


\begin{card}
	Werten Sie folgende $\lambda$-Ausdrücke aus:
	\begin{enumerate}[a)]
	\item $((\lambda x.\lambda y.(y \quad x) \quad \lambda p.\lambda q.p) \quad \lambda i.i)$
	\item $((\lambda x.\lambda y.\lambda z((x \quad y) \quad z) \quad \lambda f.\lambda a.(f \quad a)) \quad \lambda i.i) \quad\lambda j.j)$
	\end{enumerate}
	\hr
	Der \underline{Ausdruck} wird als Wert in dem/der \textbf{Symbol/Variable} ersetzt, festgelegt durch $\lambda\mathbf{Symbol}$. Beachte die Klammern um den jeweilige funktion, die erst die Auswertung erlaubt.
	\begin{enumerate}[a)]
	\item
	$((\lambda \mathbf{x}.\lambda y.(y \quad \mathbf{x}) \quad \uline{\lambda p.\lambda q.p}) \quad \lambda i.i) 
	\Rightarrow 
	((\lambda y.(y \quad \lambda p.\lambda q.p)) \quad \lambda i.i)$\\
	
	$((\lambda \mathbf{y}.(\mathbf{y} \quad \lambda p.\lambda q.p)) \quad \uline{\lambda i.i})
	\Rightarrow
	(\lambda i.i \quad \lambda p.\lambda q.p)$\\
	
	$(\lambda \mathbf{i}.\mathbf{i} \quad \uline{\lambda p.\lambda q.p})
	\Rightarrow
	\lambda p.\lambda q.p$
	\item Lösungsweg fehlt: $\lambda j.j$
	\end{enumerate}	
\end{card}


\begin{card}
	Wieso wird das abstrakteste $\lambda$-Kalkül als \textbf{typfrei} bezeichnet?
	\hr
	Arbeit nur auf Symbolen, reine Textersetzung
\end{card}

\begin{card}
	Bei $\lambda$-Kalkül–Ausdrücken wird von Auswertung und Abstraktion gesprochen. Erklären Sie an einem Beispiel, in wiefern bei $\lambda$-Kalkül–Ausdrücken abstrahiert und	konkretisiert wird.
	\hr
	
\end{card}

\begin{card}
	Mit welchen Programmierkonzept aus C ist das typfreie $\lambda$-Kalkül vergleichbar? 
	\begin{enumerate}[a)]
	\item Makros
	\item Templates
	\item Funktionen
	\item Pointern
	\end{enumerate}
	\hr
\end{card}

\begin{card}
	Im Zusammenhang mit der Auswertung von $\lambda$-Ausdrücken kann es zu Namenskonflikten	kommen, die mit Hilfe der sogenannten
	$\alpha$
	Konvertierung gelöst werden. Erklären Sie an einem Beispiel die Problematik. Wie wird das Problem konkret gelöst?
	\hr
\end{card}

\begin{card}
	Die Auswertung von Ausdrücken wird als $\beta$–Reduktion bezeichnet. Welches Problem tritt hier auf? (Beispiel!)
	\hr
\end{card}

\begin{card}
	Die einzige Möglichkeit im $\lambda$-Kalkül eine Wiederholung zu formulieren basiert auf dem sogenannten Fixpunktsatz. Was ist damit gemeint? Wieso lösen Fixpunkte das Problem der Wiederholung? 
	\hr
	Tafelanachrieb?
\end{card}