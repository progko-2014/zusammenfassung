\begin{card}
	\frametitle{Übung 3: Paradigmen}
	\url{http://people.f4.htw-berlin.de/~hebold/htw/pka/exercises/konzepte-Paradigmen.pdf}
\end{card}

\begin{card}
	Das von-Neumann-Rechnerkonzept (auch von-Neumann-Architektur) zählt zur archetypischen Realisierung des imperativen Programmierparadigmas. Warum?
	\hr
	Imperative Konzept $\ent$ Befehlsorientiert\\
	Fetch, Execute-Zyklus
\end{card}

\begin{card}
Die Turing-Maschine realisiert ebenfalls das imperativen Programmierparadigma. Warum?
\hr
Jeder Zustand verknüpft über Befehle, vgl. Überführungsfunktion
\end{card}

\begin{card}
	Wiese wird vom von-Neumann-Rechner\textbf{konzept} aber von der Turing-\textbf{Maschine} gesprochen?
	\hr
	Konzept: Abstraktion
	
	Maschine: Konkrete Idee (auch wenn so nicht realisierbar)
	\begin{figure}[h]
	\centering
	\includegraphics[width=4cm]{Von-Neumann_Architektur}
	\includegraphics[width=6cm]{Turingmaschine}
	\caption{Von-Neumann, Turing-Maschine}
	\end{figure}
\end{card}

\begin{card}
	Im Zusammenhang mit dem Neumann-Rechnerkonzept ist die Rede vom von-Neumann-Flaschenhals, wenn Nachteile des Konzepts genannte werden.
	
	\hr
	a) Was ist darunter zu verstehen?
	
	Alle Befehle müssen durch den Bus
	\hr
	b) Gibt es eine vergleichbare Problematik für die Turing-Maschine? 
	
	Schreib/Lesekopf kann nur entweder schreiben oder lesen
\end{card}

\begin{card}
	Nennen Sie wenigstens einen konzeptionellen Unterschied zwischen von-Neumann-Rechnerkonzept und Turing-Maschine.
	\hr
	Von TM ausgehend:
	
	\begin{enumerate}
	\item Daten und Programme liegen \textbf{nicht} im selben Speicher
	\item keine Nummerierung auf dem Band
	\item keine Sprungadressen
	\item kann nur 1 Feld gehen pro Befehl
	\end{enumerate}
\end{card}

\begin{card}
	Setzt die Turing-Maschine das von-Neumann-Rechnerkonzept um?
	\hr
	\textbf{Nein}, weil
	\begin{enumerate}
	\item Bei TM: Daten $\neq$ Programme
	\item TM hat keine Sprungadresse
	\end{enumerate}
	oder \textbf{Ja} mit Einschränkungen (s.o)
\end{card}

\begin{card}
	Wie könnte das Paradigma der strukturierten Programmierung in das von-Neumann-Rechnerkonzept integriert werden?
	\hr
	Überwachen, bzw. Regeln der Sprunganweisungen.\\
	D.h. Begrenzter Bereich z.B. bei if-Anweisungen
\end{card}

\begin{card}
	Wieso verletzt das Konzept der lokalen static-Variablen in C das Paradigma der funktionalen
	Programmierung?
	\hr
	Pardigma der f. Programmierung: Funktionsausgabe nur abhängig von Eingabe. D.h. bei gleicher Eingabe gleiche Ausgabe.
	\begin{lstlisting}[language=C]
int f(int i) {
  static int x = 0; 
  // Ausfuehrung bei Objekt-Init, nicht bei Methodenaufruf
  x++;
  return x+i;
}
	\end{lstlisting}	
\end{card}

\begin{card}
	Wieso verletzen Pointer in C das Paradigma der funktionalen Programmierung?
	\hr
	Pardigma der f. Programmierung: Funktionsausgabe nur abhängig von Eingabe.  D.h. bei gleicher Eingabe gleiche Ausgabe.
	\begin{lstlisting}[language=C]
int f(int *i) {
  *i = 1234 // Veraendern der Speicheradresse und somit der Eingabe
  ...
}
	\end{lstlisting}	
\end{card}

\begin{card}
	In Java gibt es mit dem Collection-Framework eine Reihe von sogenannten Container-Klassen. Welches objektorientierte Programmierparadigma verletzen Objekte zB. der Klassen	ArrayList oder Vector? 
	\hr
	
\end{card}