\begin{card}
	\frametitle{Übung 7: Komplexitätsklassen}
	\url{http://people.f4.htw-berlin.de/~hebold/htw/pka/exercises/komplexit\%C3\%A4t.pdf}
\end{card}

\begin{card}
	Die Komplexitätsklasse $\mathbf{P}$ wird üblicherweise als Entscheidungsproblem definiert.
	\begin{enumerate}[a)]
	\item Formulieren Sie die entsprechende Definition.
	\item Formulieren Sie $\mathbf{P}$ als Suchproblem.
	\end{enumerate}
	\hr
\end{card}

\begin{card}
	Die Komplexitätsklasse $\mathbf{NP}$ wird üblicherweise als Entscheidungsproblem definiert.
	\begin{enumerate}[a)]
	\item Formulieren Sie die entsprechende Definition.
	\item Formulieren Sie $\mathbf{NP}$ als Suchproblem. 
	\end{enumerate}
	\hr
\end{card}

\begin{card}
	Die Zeitkomplexität eines Algorithmus wird in Abhängigkeit von der Länge der Eingabe auf der Turing-Maschine gemessen und nicht in Abhängigkeit vom Wert der Eingabe. Dabei bleibt die entsprechende Funktion $time_F$, die die Zeitkomplexität von Algorithmus F beschreibt, auf [01]* definiert, dh.
	$time_F:[01]^* \rightarrow \mathbb{N}_0 : x \mapsto max \{ steps_F(y):|x| = |y| \}$
	mit $steps_F(y)$ als Funktion, die die Rechenschritte zu y liefert. 
	\begin{enumerate}[a)]
	\item Nennen Sie zwei Gründe für diesen Wechsel des Maßes. Mit
	$time_F$ wird die folgende Relation definiert: 
	$x \sim y := time_F(x) = time_F(y)$
	\item Beschreiben Sie verbal, was die Relation besagt.
	\item Prüfen Sie, ob die Relation die drei klassischen Eigenschaften einer Relation erfüllt, dh. prüfen Sie, ob $\sim$ eine Äquivalenzrelation ist. 
	\end{enumerate}
	\hr
\end{card}

\begin{card}
	Was besagen die Aussagen: 
	\begin{enumerate}[a)]
	\item $\mathbf{P = NP}$
	\item $\mathbf{P \subset NP}$
	\item $\mathbf{NP \subset P}$
	\end{enumerate}
	in der Definition mit \textbf{Turing-Maschinen} und \textbf{Beweissystemen}
	\hr
\end{card}

\begin{card}
	Lösungsfunktion und Algorithmus sind etwas anderes.
	\begin{enumerate}[a)]
	\item Erklären Sie den Unterschied an einem Beispiel.
	\item Wieso beziehen sich die Begriffe	\textbf{P} und \textbf{NP} auf Algorithmen und nicht auf Funktionen ? 
	\end{enumerate}
	\hr
	\begin{enumerate}[a)]
	\item Primzahl und Verfahren zur Berechnung, ob eine Primzahl ist oder nicht.
	\item Wieso beziehen sich die Begriffe	\textbf{P} und \textbf{NP} auf Algorithmen und nicht auf Funktionen ? 
	\end{enumerate}
	\hr
\end{card}

\begin{card}
	In welche der beiden folgenden Komplexitätsklassen gehört ein Programm, das aus einer
	gegebenen endlichen Menge alle Teilmengen erzeugt?
	\begin{enumerate}[a)]
	\item \textbf{P}
	\item \textbf{NP}
	\end{enumerate}
	\hr
\end{card}

\begin{card}
	\begin{columns}
		\begin{column}{.5\linewidth}
		Bestimmen Sie die Ableitungen der folgenden Funktionen:
			\begin{enumerate}[a)]
			\item $f(x) = x * ln~x$
			\item $f(x) = ln ( ln~x)$
			\item $f(x) = log_n x$
			\item $f(x) = lg~x$
			\item $f(x) = x^x$
			\item $f(x) = e^{ln~x}$
			\end{enumerate}
		\end{column}
		\begin{column}{.5\linewidth}
		Lösung:
			\begin{enumerate}[a)]
			\item $f'(x) = ln~x$ mit P.
			\item $f'(x) = \frac{1}{x*ln~x}$ mit K. + P.
			\item $f'(x) = \frac{1}{x*ln~n}$ mit P.
			\item $f'(x) = \frac{1}{x*ln~10}$
			\item $f'(x) = ln~x*x^x$ mit K. + P.
			\item $f'(x) = 1 (= 1 * x^{1-1})$
			\end{enumerate}
		\end{column}
	\end{columns}
	\vfill	
	Legende: Anwendung der Kettenregel (K), Produktregel (P)\\
	Hinweise:\\
	\begin{itemize}
	\item $lg = log_{10}$, $ln = log_e$,
	\item $e^{ln(x)} = x \text{, da aus } a^b = c \text{ und } b = log_a~c \text{ folgt } a^{b = log_a~c} = c$
	\item $log_a~b = \frac{ln~b}{ln~a}$
	\item $\frac{a}{b} = a * b^{-1}$
	\end{itemize}
\end{card}

\begin{card}
	Finden Sie zu jeder der gegebenen Funktionen $f$ eine Funktion $g$, so dass $f \in \mathbf{O}(g)$ gilt:
	\begin{columns}
		\begin{column}{.5\linewidth}
			\begin{enumerate}[a)]
			\item $f(n) =f(n - 1 ) + 10$
			\item $f(n) =f(n- 1 ) +n$
			\item $f(n) = 2*f(n- 1 )$
			\item $f(n) =f(n/ 2 ) + 10$
			\item $f(n) =f(n/ 2 ) +n$
			\item $f(n) = 2*f(n/ 2 ) +n$
			\item $f(n) = 3*f(n/ 2 )$
			\item $f(n) = 2*f(n/ 2 ) + \mathbf{O} (n2)$
			\end{enumerate}
		\end{column}
		\begin{column}{.5\linewidth}
			\begin{enumerate}[a)]
			\item 
			\item 
			\item 
			\item 
			\item 
			\item 
			\item 
			\item 
			\end{enumerate}
		\end{column}
	\end{columns}
\end{card}

\begin{card}
	Nennen Sie jeweils zwei Beispiele für Probleme, die von:
	\begin{enumerate}[a)]
	\item konstanter
	\item logarithmischer
	\item linearer
	\item quadratischer
	\item polynomieller
	\item exponentieller
	\item faktorieller 
	\end{enumerate}
	Wachstumsordnung sind.
	\hr
\end{card}

\begin{card}
	Beschreiben Sie verbal oder auch formal die Bedeutung der Ausdrücke: 
	\begin{enumerate}[a)]
	\item $f \in \mathbf{O}(n^2)$
	\item $\mathbf{O}(f) = \mathbf{O}(n^2)$
	\item $\mathbf{O}(f) \in \mathbf{O}(n^2)$
	\item $n^2 \in \mathbf{O}(e^x)$
	\item $f(x) = 1+x^2+\mathbf{O}(log x)$
	\item $\mathbf{O}(f) \subseteq \mathbf{O}(n^2)$
	\end{enumerate}
	Hinweis: Falls ein Ausdruck sinnlos ist, sollten Sie das vermerken.
	\hr
	\begin{enumerate}[a)]
	\item $f$ ist Element der Menge aller Funktionen mit quadratischem Wachstum
	\item Wachstumsklasse $f$ ist äquivalent zur Wachstumsklasse $n^2$
	\item \lightning  $\mathbf{O}(n^2)$ enthält keine Mengen
	\item existiert ein c für: $n^2 \leq c * e^x$ ?
	\item Formal: $=$ macht keinen Sinn, $\in$ wäre hier richtig
	\item Teilmenge kann es sein, vgl. c)
	\end{enumerate}
\end{card}

\begin{card}
	Die Symbole $\mathbf{O}, \Omega, \Theta,o \text{ und } \omega$ beschreiben Relationen zwischen Funktionen.
	\begin{enumerate}[a)]
	\item Formulieren Sie die Aussagen, die erfüllt sein müssten, damit die genannten Relationen reflexiv, symmetrisch und transitiv sind.\\
	(Hinweis: Es geht ausdrücklich nicht darum, eine gültige Aussage zu formulieren,
	sondern nur um die Formalisierung.)
	\item Überprüfen Sie die Gültigkeit der Aussagen aus a). 
	\item Handelt es sich bei den genannten Relationen jeweils um eine:
	Äquivalenzrelation, Quasiordnung, Ordnung und/oder Halbordnung 
	\end{enumerate}
\end{card}