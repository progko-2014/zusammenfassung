\begin{card}
	\frametitle{Übung 7: Komplexitätsklassen}
	\url{http://people.f4.htw-berlin.de/~hebold/htw/pka/exercises/komplexit\%C3\%A4t.pdf}
\end{card}

\begin{card}
	Die Komplexitätsklasse $\mathbf{P}$ wird üblicherweise als Entscheidungsproblem definiert.
	\begin{enumerate}
	\item Formulieren Sie die entsprechende Definition.
	\item Formulieren Sie $\mathbf{P}$ als Suchproblem. 
	\end{enumerate}
	\hr
\end{card}

\begin{card}
	Die Komplexitätsklasse $\mathbf{NP}$ wird üblicherweise als Entscheidungsproblem definiert.
	\begin{enumerate}
	\item Formulieren Sie die entsprechende Definition.
	\item Formulieren Sie $\mathbf{NP}$ als Suchproblem. 
	\end{enumerate}
	\hr
\end{card}

\begin{card}
	Die Zeitkomplexität eines Algorithmus wird in Abhängigkeit von der Länge der Eingabe auf	der Turing-Maschine gemessen und nicht in Abhängigkeit vom Wert der Eingabe. Dabei bleibt die entsprechende Funktion $time_F$, die die Zeitkomplexität von Algorithmus F beschreibt, auf [01]* definiert, dh. 
	
	\hr
\end{card}